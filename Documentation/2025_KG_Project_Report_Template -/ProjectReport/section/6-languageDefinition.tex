\section{Language Definition}
This section describes the Language Definition phase. The main objective of this phase is to ensure clarity, consistency, and interoperability by addressing the ambiguity and diversity of both natural and domain-specific languages. By fixing the vocabulary and formalizing the meaning of concepts, object properties, and data properties, this phase enables accurate data annotation, reduces misunderstandings, and supports system integration. The outputs produced in this phase are stored in the folder: Phase 2 - Information Gathering.

\subsection{Concept Identification}
During the language definition phase, the relevant concepts needed to represent the information defined in the Purpose Definition phase were identified. Each concept was assigned a unique identifier ID and an accompanying gloss to clarify its meaning. This approach ensures reusability and reduces ambiguity. Some examples of the defined concepts are reported in Table~\ref{tab:language_resources}.

\begin{table}[h!]
    \centering
    \begin{tabular}{|l|l|l|l|}
        \hline
        \textbf{ID} & \textbf{Custom Concept} & \textbf{Equivalent Concept - Knowledge Source} & \textbf{Gloss} \\
        \hline
        KG-25-1 & WeatherReport & WeatherReport - Home Weather Ontology & Daily meteorological report \\
        KG-25-2 & Date & Date - Schema.org & A date value in ISO 8601 format \\
        \hline
    \end{tabular}
    \caption{Examples of concepts defined during the Language Definition phase.}
    \label{tab:language_resources}
\end{table}
Concepts were aligned with the Language Teleontology when available. When no appropriate parent concept existed, the Teleontology was enriched with project-specific concepts to maintain completeness and consistency with the project purpose. In some cases, when formal descriptions were not available in reference sources, custom definitions were created to ensure that all necessary concepts were properly represented.


%DATA LAYER, DATASET CLEANING
\subsection{data layer}
        \begin{itemize}
            \item Dataset filtering/cleaning/formatting.
        \end{itemize}


\noindent The report of the work done during this phase of the methodology, has to includes also the description of the  different choices made, with their strong and weak points. In other words the report should provide to the reader, a clear description of the reasoning conducted by all the different team members.

Evaluation - Language Definition: A detailed description of the Language layer evaluation over the language layer of the KG -
\begin{itemize}
    \item How many etypes terms initially considered? How many etypes terms finally considered? How many etypes terms could be aligned to the  language teleontology? report each etype term level details in a table.
    \item How many object property terms initially considered? How many object property terms finally considered?  How many object property terms could be aligned to the  language teleontology? report each object property term level details in a table.
    \item How many data property terms initially considered? How many data property terms finally considered?  How many data property terms could be aligned to the language teleontology? report each data property term level details in a table.
    \item If valid, how was the language teleontology enriched/adapted? report element level details in a table.
    \item Did you have to return to change something in the Purpose Definition phase? If yes, report here.
    \item If valid, report dataset-level formatting and transformations done in this phase?
\end{itemize}
