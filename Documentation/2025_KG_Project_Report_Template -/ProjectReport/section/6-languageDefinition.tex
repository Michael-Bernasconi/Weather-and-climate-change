\section{Language Definition}
This section describes the Language Definition phase. The main objective of this phase is to ensure clarity, consistency, and interoperability by addressing the ambiguity and diversity of both natural and domain-specific languages. By fixing the vocabulary and formalizing the meaning of concepts, object properties, and data properties, this phase enables accurate data annotation, reduces misunderstandings, and supports system integration. The outputs produced in this phase are stored in the folder: Phase 2 - Language definition.

\subsection{Concept Identification}
During the Language Definition phase, the relevant concepts required to represent the information defined in the Purpose Definition phase were identified. Each concept was assigned a unique identifier (ID) and an accompanying gloss to clarify its meaning. This approach ensures reusability and reduces ambiguity. Examples of the defined concepts are reported in Figure \ref{fig:language_definition}.

\begin{figure}[h]
    \centering
    \includegraphics[width=0.75\textwidth]{../../../../Phase 2 - Language Definition/language_definition.png}
    \caption{Classes of the ontology properly defined. Some classes are directly derived from the Weather Ontology schema, while others were created to adapt the ontology to the specific domain.}
    \label{fig:language_definition}
\end{figure}

Concepts were aligned with the Language Teleontology whenever possible. In cases where no appropriate parent concept existed, the Teleontology was extended with project-specific concepts to maintain completeness and consistency with the project objectives. For concepts lacking formal descriptions in reference sources, custom definitions were created to ensure comprehensive representation.

\subsection{Knowledge Layer}
The Knowledge Layer defines the classes of the Knowledge Graph. A reuse strategy was applied to incorporate as many pre-existing classes as possible. The foundational ontology employed is the Home Weather Ontology (HWO), as indicated in the Information Gathering section. This ontology was selected due to the large number of domain-relevant concepts it provides.  

After verifying the existing classes, additional eTypes were defined to cover domain-specific requirements (e.g., the \textit{Anomaly} concept, relevant for Scenario 3). The resulting classes are shown in Figure \ref{fig:ontology_classes}, and the corresponding ontology file is available \href{https://github.com/Michael-Bernasconi/Weather-and-climate-change/blob/56b059a57c71f5885cc68e8ff9bd61367d3b069b/Phase%202%20-%20Language%20Definition/language_definition_ontology.owx}{here}.

\begin{figure}[h]
    \centering
    \includegraphics[width=0.5\textwidth]{../../../../Phase 2 - Language Definition/ontology_definition.png}
    \caption{Classes of the ontology properly defined. Some classes were imported from existing ontologies, while others were defined from scratch.}
    \label{fig:ontology_classes}
\end{figure}

The Home Weather Ontology already included key concepts such as \textit{Weather Report}. Additional properties were defined to align these concepts with the domain-specific requirements. The concept \textit{WeatherReportSource} was retained, and \textit{WeatherStation} was defined as a subclass of this concept. Classes representing temporal instants and intervals were imported from the W3.org ontology.

\subsection{Knowledge-Data Link}

The phase following the Knowledge Layer definition establishes an explicit link between ontology concepts (classes and properties) and the corresponding data sources. This linkage ensures that the \textbf{Knowledge Graph (KG)} can be instantiated with real and up-to-date data.

The \textbf{Knowledge-Data link} was established by mapping each \textbf{Custom Concept} in the ontology to specific \textbf{Data Concepts} available in the selected sources. The primary data sources used were:

\begin{itemize}
    \item \textbf{External APIs (API):} Providing real-time or historical data for fundamental meteorological measurements such as temperature ($MeanTemp$, $MaxTemperature$, $MinTemperature$), precipitation ($Precipitation$), humidity, and wind.
    \item \textbf{Open Data Trentino:} Offering descriptive and geographical information regarding weather stations ($WeatherStation$, $Name$, $Elevation$, $Latitude$, $Longitude$) and territorial entities ($City$, $Shortname$, $Region$).
\end{itemize}

The mapping process identified, for each key concept in the Knowledge Graph (e.g., $WeatherReport$, $WeatherStation$, $Anomaly$), the specific attributes required for its representation (e.g., $Date$, $WeatherCode$, $StartDate$, $EndDate$) and the corresponding data sources. The outcome of this process is shown in Figure \ref{fig:knowledge_data_link}.

Additionally, the \textbf{Knowledge-Data link} includes generated concepts (e.g., $TypeAnomaly$, $MicroClimate$, $ClimateTrend$, $Season$, and $Variation$), which do not originate directly from API endpoints but result from internal processing or inference applied to the raw data. This approach ensures that the Knowledge Graph functions as a repository of both raw and processed knowledge.

\centerline{\includegraphics[width=0.8\textwidth]{../../../../Phase 2 - Language Definition/knowledge_data_link.png}}
\captionof{figure}{For each ontology concept, the corresponding attribute extracted from the data sources is indicated.}
\label{fig:knowledge_data_link}
\vspace{1em}

\subsection{Data Layer}
The Data Layer is responsible for acquiring, filtering, cleaning, and formatting the datasets used within the system.

\subsubsection{Data Collected from IlMeteo.it}
Historical weather data were obtained from \textit{IlMeteo.it}, organized by month and year, spanning from January 2000 to October 2025. Data were collected for the following locations: Rovereto, Trento, Povo, Tenno, Mezzana, Predazzo, Lavarone, Telve, Cavalese, and Arco. Each file contained complete daily weather observations for a specific month and location.

During preprocessing, only relevant columns were retained to standardize the format and remove redundant information. Selected attributes include: 
\begin{itemize}
    \item \textbf{Location}
    \item \textbf{Date}
    \item \textbf{MeanTemp}
    \item \textbf{MinTemperature}
    \item \textbf{MaxTemperature}
    \item \textbf{MeanHumidity}
    \item \textbf{WindSpeed}
    \item \textbf{WindGusts}
    \item \textbf{Rainfall}
    \item \textbf{Phenomena}
\end{itemize}

Datasets were merged chronologically to maintain a consistent structure suitable for integration into the Knowledge Graph. All files are stored in the folder \textit{Phase 2 - Language Definition}.

\subsubsection{Data Collected from APIs}
In addition to historical datasets, data were obtained from external APIs providing current weather parameters such as temperature, humidity, wind speed, and precipitation in machine-readable formats (e.g., CSV). These datasets were cleaned and normalized to maintain consistency with the IlMeteo.it data, ensuring seamless integration into the data layer.

\subsubsection{Data Collected and Derived from APIs and Local Sources}
The Data Layer also integrates meteorological data obtained from APIs. These datasets, supplied in machine-readable formats, were subjected to cleaning and normalization to ensure structural and semantic consistency with historical datasets.

High-level entities were derived from raw and aggregated data. These derived datasets provide key nodes and relationships within the Knowledge Graph:

\begin{itemize}
    \item \textbf{WeatherReport:} A chronological record of daily weather observations, obtained by merging cleaned historical data (IlMeteo.it) and real-time observations (API) to create a continuous dataset for all monitored cities (\textit{Rovereto, Trento, Povo, Tenno, Mezzana, Predazzo, Lavarone, Telve, Cavalese, Arco}).   
    \centerline{\includegraphics[width=1\textwidth]{../../../../Phase 2 - Language Definition/Dataset Screenshot/WeatherReport.png}}
    \captionof{figure}{Example rows of the \textbf{WeatherReport} dataset.}
    \label{fig:weatherreport_data_example}
    \vspace{1em}

    \item \textbf{WeatherStation:} Represents physical monitoring infrastructure. Derived from publicly available data and enriched with metadata such as \texttt{code}, \texttt{name}, \texttt{elevation}, \texttt{latitude}, and \texttt{longitude}.
    \centerline{\includegraphics[width=0.8\textwidth]{../../../../Phase 2 - Language Definition/Dataset Screenshot/WeatherStation.png}}
    \captionof{figure}{Example rows of the \textbf{WeatherStation} dataset.}
    \label{fig:weatherstation_data_example}
    \vspace{1em}

    \item \textbf{Region:} Establishes the geographic hierarchy, listing all monitored locations. Derived from the locations queried via APIs.
    \centerline{\includegraphics[width=0.1\textwidth]{../../../../Phase 2 - Language Definition/Dataset Screenshot/Region.png}}
    \captionof{figure}{Example rows of the \textbf{Region} dataset.}
    \label{fig:region_data_example}
    \vspace{1em}

    \item \textbf{Microclimate:} Classifies local climates for each city by aggregating minimum and maximum temperature values and categorizing results into \texttt{cold}, \texttt{temperate}, or \texttt{hot}.
    \centerline{\includegraphics[width=0.8\textwidth]{../../../../Phase 2 - Language Definition/Dataset Screenshot/Microclimate.png}}
    \captionof{figure}{Example rows of the \textbf{Microclimate} dataset.}
    \label{fig:microclimate_data_example}
    \vspace{1em}

    \item \textbf{Season:} Abstracts daily data into seasonal averages, facilitating multi-year comparisons and seasonal analysis.
    \centerline{\includegraphics[width=0.5\textwidth]{../../../../Phase 2 - Language Definition/Dataset Screenshot/Season.png}}
    \captionof{figure}{Example rows of the \textbf{Season} dataset.}
    \label{fig:season_data_example}
    \vspace{1em}

    \item \textbf{ClimateTrend:} Represents long-term temperature trends for each city. Produces a \texttt{Rate} (e.g., $^\circ$C/year) and classifies the trend as \texttt{Warming}, \texttt{Cooling}, or \texttt{Stable}.
    \centerline{\includegraphics[width=0.9\textwidth]{../../../../Phase 2 - Language Definition/Dataset Screenshot/ClimateTrend.png}}
    \captionof{figure}{Example rows of the \textbf{ClimateTrend} dataset.}
    \label{fig:climatetrend_data_example}
    \vspace{1em}

    \item \textbf{Anomaly:} Identifies and classifies extreme weather events by comparing observations with historical norms. Attributes include \texttt{TypeAnomaly} and qualitative \texttt{Severity} levels.
    \centerline{\includegraphics[width=0.8\textwidth]{../../../../Phase 2 - Language Definition/Dataset Screenshot/Anomaly.png}}
    \captionof{figure}{Example rows of the \textbf{Anomaly} dataset.}
    \label{fig:anomaly_data_example}
    \vspace{1em}
\end{itemize}

Following analysis of both the Knowledge and Data layers, modifications were introduced to the ER model and previously defined competency questions. The main change concerns the \textit{Region} class: as weather reports correspond to stations located in distinct cities, the concept was redefined as \textit{City}. Consequently, several competency questions were updated:

\begin{tabular}{|p{5cm}|p{1cm}|p{10cm}|}
\hline
\textbf{Person} & \textbf{N.o.} & \textbf{Competency Question (CQ)} \\
\hline
Maria Bianchi & 1.3 & How has the average monthly rainfall evolved in each \textbf{city} of Trentino over the last ten years? \\ 
Giulia Ferrari & 2.2 & Which \textbf{cities} show microclimatic differences despite geographical proximity? \\ 
Giulia Ferrari & 2.5 & How stable are microclimates across seasons (winter vs. summer) in different \textbf{cities}? \\ 
Giulia Ferrari & 2.6 & Which \textbf{cities} show the most distinctive microclimatic patterns according to sensor data? \\ 
Marco Ricci & 5.2 & How have long-term temperature and precipitation trends evolved across different \textbf{cities}? \\
\hline
\end{tabular}

Persona and Scenario 2 were also updated to reflect the change from \textit{Region} to \textit{City}, ensuring consistency with available data:

\begin{itemize}
    \item \textbf{Scenario}: A climatology student intends to study microclimates in the \textbf{different cities of Trentino}, requiring access to temperature, humidity, and wind data collected by various sensors across the territory.
    \item \textbf{Persona}: Giulia Ferrari, 23 years old, a climatology student, focusing on microclimatic and climate variations \textbf{between different cities of Trentino}.
\end{itemize}

Furthermore, the \textit{Anomaly} entity was restructured: as anomalies are strictly city-specific, the link with \textit{WeatherReport} was removed and the entity was connected directly to the city. The final ER model is shown in Figure \ref{fig:er_ld}.

\begin{figure}[h]
    \centering
    \includegraphics[width=1\textwidth]{../../../../Phase 2 - Language Definition/er_model_language_definition.png}
    \caption{ER model after the Language Definition Phase.}
    \label{fig:er_ld}
\end{figure}
