\section{Language Definition}
This section describes the Language Definition phase. The main objective of this phase is to ensure clarity, consistency, and interoperability by addressing the ambiguity and diversity of both natural and domain-specific languages. By fixing the vocabulary and formalizing the meaning of concepts, object properties, and data properties, this phase enables accurate data annotation, reduces misunderstandings, and supports system integration. The outputs produced in this phase are stored in the folder: Phase 2 - Information Gathering.

\subsection{Concept Identification}
During the language definition phase, the relevant concepts needed to represent the information defined in the Purpose Definition phase were identified. Each concept was assigned a unique identifier ID and an accompanying gloss to clarify its meaning. This approach ensures reusability and reduces ambiguity. Some examples of the defined concepts are reported in Figure \ref{fig:language_definition}

\begin{figure}[h]
    \centering
    \includegraphics[width=0.75\textwidth]{../../../../Phase 2 - Language Definition/language_definition.png}
    \caption{Classes of our ontology properly defined. There are some classed directly from the Weather Ontology schema and some defined to fit the ontology to our domain}
    \label{fig:language_definition}
\end{figure}

Concepts were aligned with the Language Teleontology when available. When no appropriate parent concept existed, the Teleontology was enriched with project-specific concepts to maintain completeness and consistency with the project purpose. In some cases, when formal descriptions were not available in reference sources, custom definitions were created to ensure that all necessary concepts were properly represented.

\subsection{Knowledge layer}
The Knowledge layer is responsible for define the classes of our Knowldege Graph. We have tried to reuse as many already defined classes as possible and, in particular, the base of our ontology is represented by Home Weather Ontology, as already indicated in the Information Gathering section of the previous chapter. The choice to use that ontology is strictly related to the fact that it has defined a lot of usefull concepts for our domain.  

Once checked the already defined classes, we started to define the eTypes that we needed in order to complete our domain (e.g. the Anomaly concept that is relevant for our Scenario 3). The final results in term of classes defined of this process is shown on Figure \ref{fig:ontology_classes}. 

\begin{figure}[h]
    \centering
    \includegraphics[width=0.5\textwidth]{../../../../Phase 2 - Language Definition/ontology_definition.png}
    \caption{Classes of our ontology properly defined. Some of the classes has been directly imported from other ontologies, while others has been defined from scratch}
    \label{fig:ontology_classes}
\end{figure}

The Home Weather Ontology had already defined one of our key concept: Weather Report. To the properties that are already defined, in the next phase of Knowledge definition it will be necessary to defined couple more to fit to our domain. We also decided to keep the concept of \textit{WeatherReportSource} and to make our \textit{WeatherStation} a descendant of that class. Also classes that define temporal istant and intervals were already present in HWO and they were imported from another ontology W3.org. 

\subsection{Knowledge - Data Link}

The subsequent phase following the definition of the Knowledge Layer is the establishment of an explicit link between the ontology concepts (the defined classes and properties) and the data sources that will populate them. This activity is crucial to ensure that the \textbf{Knowledge Graph (KG)} can be instantiated with real and up-to-date data.

The \textbf{Knowledge-Data link} was achieved by mapping each \textbf{Custom Concept} defined in our ontology to specific \textbf{Data Concepts} available through the selected data sources. The main data sources used are:

\begin{itemize}
    \item \textbf{External APIs (API):} Used to obtain real-time or historical data related to fundamental meteorological measurements such as temperature ($MeanTemp$, $MaxTemperature$, $MinTemperature$), precipitation ($Precipitation$), humidity, and wind.
    \item \textbf{Open Data Trentino:} A fundamental resource for descriptive and geographical data concerning weather stations ($WeatherStation$, $Name$, $Elevation$, $Latitude$, $Longitude$) and territorial entities ($City$, $Shortname$, $Region$).
\end{itemize}

The mapping process allowed us to identify, for each key concept in the Knowledge Graph (e.g., $WeatherReport$, $WeatherStation$, $Anomaly$), the specific attributes needed for its representation (e.g., $Date$, $WeatherCode$, $StartDate$, $EndDate$) and the corresponding source from which to extract the value. The result of this process is shown in Figure \ref{fig:knowledge_data_link}

Furthermore, the \textbf{Knowledge-Data link} includes concepts that have been \textbf{Generated} (such as $TypeAnomaly$, $MicroClimate$, $ClimateTrend$, $Season$, and $Variation$), which do not originate directly from an API endpoint but are the result of internal processing or inference processes applied to the raw data extracted from the primary sources. This approach ensures that the Knowledge Graph is not merely a data container but also a hub for \textbf{processed knowledge}.

% Inserimento Immagine forzato
    \centerline{\includegraphics[width=0.8\textwidth]{../../../../Phase 2 - Language Definition/knowledge_data_link.png}}
    \captionof{figure}{For each concept of the ontology, it's indicated the relative attribute taken in the data }
    \label{fig:knowledge_data_link}
    \vspace{1em}

\subsection{Data Layer}
The data layer is responsible for acquiring, filtering, cleaning, and formatting the datasets used in the system.
\subsubsection{Data Collected from IlMeteo.it}

Historical weather data were downloaded from \textit{IlMeteo.it}, organized by month and year, from January 2000 to October 2025. 
The datasets were collected for the following locations: Rovereto, Trento, Povo, Tenno, Mezzana, Predazzo, Lavarone, Telve, Cavalese, and Arco. 
Each file contained a complete set of daily weather observations for a given month and location.

During the data cleaning and preprocessing phase, only the relevant columns were retained in order to standardize the format and remove redundant information. 
The selected attributes were: 
\begin{itemize}
    \item \textbf{Location}
    \item \textbf{Date}
    \item \textbf{MeanTemp}
    \item \textbf{MinTemperature}
    \item \textbf{MaxTemperature}
    \item \textbf{MeanHumidity}
    \item \textbf{WindSpeed}
    \item \textbf{WindGusts}
    \item \textbf{Rainfall}
    \item \textbf{Phenomena}
\end{itemize}

Each dataset was then merged chronologically, maintaining a consistent structure suitable for further integration into the knowledge graph. 
All files can be found in the folder Phase 2 - Language Definition.
\\

\subsubsection{Data Collected from APIs}
In addition to the historical datasets from IlMeteo.it, data were also obtained via external APIs.
These APIs provide current weather parameters such as temperature, humidity, wind speed, and precipitation in a machine-readable format (typically CSV).
The collected data undergo a similar cleaning and normalization process to ensure consistency with the IlMeteo.it datasets, enabling seamless integration within the data layer and downstream processing components of the system.

\subsubsection{Data Collected and Derived from APIs and Local Sources}
In parallel with the historical collection, the data layer integrates meteorological data obtained via external APIs. These APIs supply current weather parameters—such as temperature, humidity, wind speed, and precipitation—in a machine-readable CSV format. This data is subject to rigorous cleaning and normalization to ensure structural and semantic consistency with the \textit{IlMeteo.it} datasets, enabling seamless integration within the system.

A crucial function of the data layer is the derivation of complex, high-level entities from the raw and aggregated data. These derived datasets form essential nodes and relationships within the Knowledge Graph, providing specialized context and analytical depth:

\begin{itemize}
\item \textbf{WeatherReport}: This is the primary dataset, acting as a chronological record of daily weather observations. It is derived by merging the cleaned historical data (\textit{IlMeteo.it}) and the real-time observations (API) to create a continuous and comprehensive view of meteorological events. This comprehensive dataset is logically partitioned and organized for all ten monitored cities (\textit{Rovereto, Trento, Povo, Tenno, Mezzana, Predazzo, Lavarone, Telve, Cavalese, and Arco}).   
\centerline{\includegraphics[width=1\textwidth]{../../../../Phase 2 - Language Definition/Dataset Screenshot/WeatherReport.png}}
    \captionof{figure}{Example rows of the \textbf{WeatherReport}.}
    \label{fig:weatherreport_data_example}
\vspace{1em}

    \item \textbf{WeatherStation}: This entity maps the physical infrastructure used for data collection. It is derived from publicly available data (e.g., from the "Weather Trentino" service) and enriched with key metadata such as \texttt{code}, \texttt{name}, \texttt{elevation}, and geographic coordinates (\texttt{latitude}, \texttt{longitude}).
    % Inserimento Immagine forzato
    \centerline{\includegraphics[width=0.8\textwidth]{../../../../Phase 2 - Language Definition/Dataset Screenshot/WeatherStation.png}}
    \captionof{figure}{Example rows of the \textbf{WeatherStation} derived dataset, showing key metadata for the physical monitoring infrastructure.}
    \label{fig:weatherstation_data_example}
    \vspace{1em}

    \item \textbf{Region}: This dataset establishes the geographic hierarchy, listing all relevant locations (cities) for which weather data is tracked. It is directly derived from the list of locations queried via the APIs.
    % Inserimento Immagine forzato
    \centerline{\includegraphics[width=0.1\textwidth]{../../../../Phase 2 - Language Definition/Dataset Screenshot/Region.png}}
    \captionof{figure}{Example rows of the \textbf{Region} derived dataset, detailing the monitored locations.}
    \label{fig:region_data_example}
    \vspace{1em}

    \item \textbf{Microclimate}: This dataset provides a specialized classification of the local climate for each city. It is derived from long-term API data by aggregating minimum and maximum temperature values and classifying the result into custom categories such as \texttt{cold}, \texttt{temperate}, or \texttt{hot}, based on defined thresholds.
    % Inserimento Immagine forzato
    \centerline{\includegraphics[width=0.8\textwidth]{../../../../Phase 2 - Language Definition/Dataset Screenshot/Microclimate.png}}
    \captionof{figure}{Example rows of the \textbf{Microclimate} derived dataset, showing the temperature range and climate classification for each city.}
    \label{fig:microclimate_data_example}
    \vspace{1em}

    \item \textbf{Season}: This dataset abstracts daily weather data into seasonal averages. By aggregating temperature and precipitation values over defined seasonal periods for each year, it facilitates multi-year comparison and seasonal impact analysis.
    % Inserimento Immagine forzato
    \centerline{\includegraphics[width=0.5\textwidth]{../../../../Phase 2 - Language Definition/Dataset Screenshot/Season.png}}
    \captionof{figure}{Example rows of the \textbf{Season} derived dataset, with averaged meteorological parameters for each city and year.}
    \label{fig:season_data_example}
    \vspace{1em}

\item \textbf{ClimateTrend}: This entity is the result of a statistical time series analysis. It tracks the long-term temperature change for each city.The analysis produces a specific \texttt{Rate} (e.g., $^\circ$C/year) and classifies the overall trend as \texttt{Warming}, \texttt{Cooling}, or \texttt{Stable}.    \centerline{\includegraphics[width=0.9\textwidth]{../../../../Phase 2 - Language Definition/Dataset Screenshot/ClimateTrend.png}}
    \captionof{figure}{Example rows of the \textbf{ClimateTrend} derived dataset, detailing the calculated long-term temperature rate and variation.}
    \label{fig:climatetrend_data_example}
    \vspace{1em}

    \item \textbf{Anomaly}: This dataset identifies and classifies extreme weather events. It is derived by comparing recorded observations against established historical norms for a specific location. Each entry includes the \texttt{TypeAnomaly} (e.g., Excessive Wind, Too Hot Temperature) and a qualitative \texttt{Severity} level (e.g., Critical, High).
    % Inserimento Immagine forzato
    \centerline{\includegraphics[width=0.8\textwidth]{../../../../Phase 2 - Language Definition/Dataset Screenshot/Anomaly.png}}
    \captionof{figure}{Example rows of the \textbf{Anomaly} derived dataset, showing the type and severity of detected extreme weather events.}
    \label{fig:anomaly_data_example}
    \vspace{1em}
\end{itemize}

After working on both the knowledge layer and the data layer we have made some changes in the ER model and the competency questions precedently defined. The main change it's related to the \textit{Region} class: since the data selected in the data cleaning phase of weather reports are related to weather stations located in different cities, the concept of \textit{Region} has been properly redefined as \textit{City}. We considered this definition more appropriate to the data available. In order to reflect this change some competency questions have been slightly modified: 

\begin{tabular}{|p{5cm}|p{1cm}|p{10cm}|}
\hline
\textbf{Person} & \textbf{N.o.} & \textbf{Competency Question (CQ)} \\
\hline
Maria Bianchi & 1.3 & How has the average monthly rainfall evolved in each \textbf{city} of Trentino over the last ten years? \\ Giulia Ferrari & 2.2 & Which \textbf{cities} show microclimatic differences despite geographical proximity? \\ Giulia Ferrari & 2.5 & How stable are microclimates across seasons (winter vs. summer) in different \textbf{cities}? \\ Marco Ricci & 5.2 & Ho have long-term temperature and precipitation trends evolved across different \textbf{cities}? \\
\hline
\end{tabular}

In addition, to reflect in the best way the changes made, we also modified Persona and Scenario 2, the ones regarding Giulia. Since we don't have information about valleys and these types of areas and since we decide to change the \textit{Region} entity in the \textit{City} entity, we decided to modify them like it follows:
\begin{itemize}
    \item \textbf{Scenario}: Giulia, a climatology student, wants to stuy the microclimate present in the \textbf{different cities of Trentino}, She needs access to temperature, humidity and wind data collected by various sensors across territory.
    \item \textbf{Persona}: Giulia Ferrari, 23 years old, a climatology student. She enjoys studying microclimates and climate variations \textbf{between different cities of Trentino}.
\end{itemize}

Another modified made to the previous ER model it is about the \textit{Anomaly} entity: since an anomaly it's strictly related to the city where it has been relevated, we decided to add directly this information as an attribute of the entity. Given the modifies described, the ER model after this phase is shown on Figure \ref{fig:language_definition}

\begin{figure}[h]
    \centering
    \includegraphics[width=0.75\textwidth]{../../../../Phase 2 - Language Definition/er_model_language_definition.png}
    \caption{ER model after the Language Definition Phase}
    \label{fig:er_ld}
\end{figure}