\section{Language Definition}
This section describes the Language Definition phase. The main objective of this phase is to ensure clarity, consistency, and interoperability by addressing the ambiguity and diversity of both natural and domain-specific languages. By fixing the vocabulary and formalizing the meaning of concepts, object properties, and data properties, this phase enables accurate data annotation, reduces misunderstandings, and supports system integration. The outputs produced in this phase are stored in the folder: Phase 2 - Information Gathering.

\subsection{Concept Identification}
During the language definition phase, the relevant concepts needed to represent the information defined in the Purpose Definition phase were identified. Each concept was assigned a unique identifier ID and an accompanying gloss to clarify its meaning. This approach ensures reusability and reduces ambiguity. Some examples of the defined concepts are reported in Figure \ref{fig:language_definition}

\begin{figure}[h]
    \centering
    \includegraphics[width=0.75\textwidth]{../../../../Phase 2 - Language Definition/language_definition.png}
    \caption{Classes of our ontology properly defined. There are some classed directly from the Weather Ontology schema and some defined to fit the ontology to our domain}
    \label{fig:language_definition}
\end{figure}

Concepts were aligned with the Language Teleontology when available. When no appropriate parent concept existed, the Teleontology was enriched with project-specific concepts to maintain completeness and consistency with the project purpose. In some cases, when formal descriptions were not available in reference sources, custom definitions were created to ensure that all necessary concepts were properly represented.

\subsection{Knowledge layer}
The Knowledge layer is responsible for define the classes of our Knowldege Graph. We have tried to reuse as many already defined classes as possible and, in particular, the base of our ontology is represented by Home Weather Ontology, as already indicated in the Information Gathering section of the previous chapter. The choice to use that ontology is strictly related to the fact that it has defined a lot of usefull concepts for our domain.  

Once checked the already defined classes, we started to define the eTypes that we needed in order to complete our domain (e.g. the Anomaly concept that is relevant for our Scenario 3). The final results in term of classes defined of this process is shown on Figure \ref{fig:ontology_classes}. 

\begin{figure}[h]
    \centering
    \includegraphics[width=0.5\textwidth]{../../../../Phase 2 - Language Definition/ontology_definition.png}
    \caption{Classes of our ontology properly defined. Some of the classes has been directly imported from other ontologies, while others has been defined from scratch}
    \label{fig:ontology_classes}
\end{figure}

\subsection{Data Layer}
The data layer is responsible for acquiring, filtering, cleaning, and formatting the datasets used in the system.
\subsubsection{Data Collected from IlMeteo.it}

Historical weather data were downloaded from \textit{IlMeteo.it}, organized by month and year, from January 2000 to October 2025. 
The datasets were collected for the following locations: Rovereto, Trento, Povo, Tenno, Mezzana, Predazzo, Lavarone, Telve, Cavalese, and Arco. 
Each file contained a complete set of daily weather observations for a given month and location.

During the data cleaning and preprocessing phase, only the relevant columns were retained in order to standardize the format and remove redundant information. 
The selected attributes were: 
\begin{itemize}
    \item \textbf{Location}
    \item \textbf{Date}
    \item \textbf{MeanTemp}
    \item \textbf{MinTemperature}
    \item \textbf{MaxTemperature}
    \item \textbf{MeanHumidity}
    \item \textbf{WindSpeed}
    \item \textbf{WindGusts}
    \item \textbf{Rainfall}
    \item \textbf{Phenomena}
\end{itemize}

Each dataset was then merged chronologically, maintaining a consistent structure suitable for further integration into the knowledge graph. 
All files can be found in the folder Phase 2 - Language Definition.
\\

\subsubsection{Data Collected from APIs}
In addition to the historical datasets from IlMeteo.it, real-time meteorological data were also obtained via external APIs.
These APIs provide current weather parameters such as temperature, humidity, wind speed, and precipitation in a machine-readable format (typically CSV).
The collected data undergo a similar cleaning and normalization process to ensure consistency with the IlMeteo.it datasets, enabling seamless integration within the data layer and downstream processing components of the system.

