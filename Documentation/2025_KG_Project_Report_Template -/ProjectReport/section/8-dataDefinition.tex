\section{Entity Definition}
\label{sec:entity_definition}

This section is dedicated to the description of the \textbf{Entity Definition} phase, the final step in the iTelos methodology. The core objective is to merge the division between knowledge and data into a unified data structure, combining the knowledge structures defined in the \textbf{Knowledge Definition} phase with the aligned datasets from the \textbf{Data Layer}. The final result is a structured \textbf{Knowledge Graph (KG)} that integrates both layers.

\subsection{Overview and Objectives}

The Entity Definition phase is the final step in the iTelos methodology, with the primary objective of merging the knowledge and data layers into a unified \textbf{Knowledge Graph}. The input for this phase consisted of the cleaned and aligned data resources (collected from IlMeteo.it, APIs, and Open Data Trentino) and the ontology created in previous stages.

The objective is to address the remaining heterogeneity of data values, ensuring that entities across datasets are uniquely identified, matched, and mapped, thereby generating the final \textbf{KG}.

The main activities performed were:
\begin{itemize}
    \item \textbf{Entity Matching \& Data Reduction:} Resolving discrepancies and aligning data values through cleaning, removal of superfluous data, and calculation of necessary fields.
    \item \textbf{Entity Identification:} Ensuring each entity instance is uniquely and consistently represented.
    \item \textbf{Entity Mapping:} Combining the ontology with the corresponding data values to generate the final Knowledge Graph in RDF-Turtle format.
\end{itemize}


\subsection{Entity Matching and Data Reduction}
\label{ssec:data_reduction}

Entity matching in this project primarily focused on \textbf{data alignment and cleansing} to ensure data quality and mitigate heterogeneity. This involved removing redundant or superfluous columns and calculating necessary geographical data.

The following transformations were performed on the raw data files to produce the final Turtle files:
\begin{itemize}
    \item \textbf{Columns Eliminated for Superfluity/Redundancy:}
    \begin{itemize}
        \item \texttt{anomaly.ttl}: 'Anomaly' column eliminated.
        \item \texttt{climatetrend.ttl}: 'ClimateTrend' column removed.
        \item \texttt{microclimate.ttl}: 'MicroClimate' column removed.
        \item \texttt{weatherstation.ttl}: Redundant columns ('name', 'east', 'north') eliminated.
    \end{itemize}
    \item \textbf{Addition of Calculated Fields:}
    \begin{itemize}
        \item \texttt{city.ttl}: Calculated geographical fields for latitude and longitude were added.
    \end{itemize}
    \item \textbf{Significant Reduction of \texttt{weatherreport.ttl}:}
    \begin{itemize}
        \item Created from \texttt{WeatherReportMinimum.csv}, columns \texttt{MinTemperature}, \texttt{MaxTemperature}, \texttt{MinHumidity}, \texttt{MaxHumidity}, and \texttt{PrecipitationHours} were \textbf{eliminated} to manage its large file size.
        \item Furthermore, the temporal scope was reduced from the original range of 1990--2025 to \textbf{2010--2025} to further manage file size and computational load.
        \item The \textbf{\texttt{StationCode}} column was maintained for the crucial linkage to the \texttt{WeatherStation} entity.
    \end{itemize}
\end{itemize}

\subsection{Entity Identification}

Once the datasets were cleaned and reduced, the focus shifted to entity identification.

\begin{itemize}
    \item \textbf{Temporal Standardization:} A critical step was the modification of the date format in all generated \texttt{.ttl} files to ensure compatibility with the \textbf{ISO 8601} standard. This format is essential for correctly identifying and querying time-bound entities.
    \item \textbf{Unique Linkage:} The inclusion of \texttt{StationCode} in the \texttt{WeatherReport} entity was key for establishing a unique identifier that allows direct connection to the corresponding \texttt{WeatherStation}.
\end{itemize}


\subsection{Entity Mapping}

\textbf{Entity Mapping} integrates the defined ontology with the cleaned data values. This activity was successfully implemented using the \textbf{Karma} tool, which facilitated the creation of the mapping models.

The figure below illustrates the mapping model for the \textbf{\texttt{WeatherReport}} entity, created using the Karma tool.
\begin{center}
    \includegraphics[width=0.9\textwidth]{/home/michael/Weather-and-climate-change/Phase 4 - Entity Definition/img/WeatherReportKarmaScreen.jpeg}
    \captionof{figure}{Mapping Model for \texttt{WeatherReport} in the Karma Tool}
    \label{fig:weatherreport_karma}
\end{center}

\subsubsection{Phase Outcomes (The Knowledge Graph)}

The output of this process is a set of finalized RDF-Turtle files (\texttt{.ttl}) for each entity type:
\begin{itemize}
    \item \texttt{anomaly.ttl}
    \item \texttt{city.ttl}
    \item \texttt{climatetrend.ttl}
    \item \texttt{microclimate.ttl}
    \item \texttt{season.ttl}
    \item \texttt{weatherstation.ttl}
    \item \texttt{weatherreport.ttl} (inserted into a \textbf{ZIP archive} due to its considerable size)
\end{itemize}
These files collectively form the final, unified \textbf{Knowledge Graph}, which is now prepared for efficient queries on the \textbf{GraphDB} platform. All generated \texttt{.ttl} files are publicly available at the following \href{https://github.com/Michael-Bernasconi/Weather-and-climate-change/tree/documentation/Phase%204%20-%20Entity%20Definition}{GitHub repository}.

\vspace{0.5cm}

\subsection{Revised Competency Questions}

To ensure the resulting Knowledge Graph remains relevant to current data constraints and user interests, several Competency Questions (\textbf{CQs}) were updated during this phase. The table below lists the original questions and their modified versions, reflecting adjusted temporal scopes or geographical foci.

\begin{table}[h!]
    \centering
    \caption{Revised Competency Questions (CQs)}
    \label{tab:cqs_revised}
    \begin{tabular}{|p{0.45\textwidth}|p{0.45\textwidth}|}
        \hline
        \textbf{Original Question} & \textbf{Modified Question} \\
        \hline
        Giulia Ferrari 2.3: How does wind speed and direction vary between the Adige Valley and surrounding mountains areas? & Giulia Ferrari 2.3: How does wind speed and direction vary across the \textbf{Trentino} region? \\
        \hline
        Francesca Romano 4.1: How have the average start and end dates of each season changed in Trentino over the past 30 years? & Francesca Romano 4.1: How have the average start and end dates of each season changed in Trentino over the past \textbf{15 years}? \\
        \hline
        Francesca Romano 4.4: How does current spring temperature compare to historical averages from the 1980s and 1990s? & Francesca Romano 4.4: How does current spring temperature compare to historical averages from \textbf{2015 to 2025}? \\
        \hline
        Marco Ricci 5.1: Which areas of Trentino have recorded the highest temperature increases over the past 50 years? & Marco Ricci 5.1: Which areas of Trentino have recorded the highest temperature increases over the past \textbf{15 years}? \\
        \hline
    \end{tabular}
\end{table}

\subsection{Decisions and Reflections}

This section summarizes the key decisions and insights from the Entity Definition phase.

\subsubsection{Strengths}
\begin{itemize}
    \item \textbf{Temporal Standardization (ISO 8601):} The adoption of the \textbf{ISO 8601} standard for all date and time attributes guaranteed semantic correctness and greatly improved the consistency and interoperability of temporal queries.
    \item \textbf{Robust Entity Linkage:} The explicit inclusion of the \texttt{StationCode} in the \texttt{WeatherReport} entity ensures a reliable link between daily reports and their respective weather stations, which is essential for geographical and relational queries.
    \item \textbf{Lean Data Structure:} The cleansing and \textbf{targeted data reduction} resulted in a leaner and more focused \textbf{KG} structure, improving query efficiency and reducing the computational load on \textbf{GraphDB}.
    \item \textbf{Process Repeatability:} The utilization of the \textbf{Karma} tool ensured that the conceptual mappings were consistently translated into the RDF-Turtle format, maintaining the integrity of the iTelos process.
\end{itemize}

\subsubsection{Weaknesses}
\begin{itemize}
    \item \textbf{Source Data Heterogeneity:} The main obstacle was the \textbf{incredible heterogeneity of the data} in the original CSV files, which required a significant \textbf{manual pre-processing effort} and extended the duration of the phase.
    \item \textbf{Trade-off on \texttt{WeatherReport} Granularity and Scope:} The decision to eliminate several columns and to reduce the temporal scope from 1990--2025 to \textbf{2010--2025} was a \textbf{necessary trade-off} to ensure the efficiency and manageability of the \textbf{KG} on \textbf{GraphDB}, but it reduced the granularity and the long-term historical depth of information available for certain potential Competency Questions (\textbf{CQs}).
\end{itemize}