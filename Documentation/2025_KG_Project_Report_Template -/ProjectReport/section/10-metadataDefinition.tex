\section{Metadata Definition}
\label{sec:metadata-definition}

In this section the report collects the definitions of all the metadata defined for the different resources produced along the whole process. The metadata defined in this phase describes both the final outcome of the project, and the intermediate outcome of each phase.

The structure of this section is organized to describe the metadata related to all types of resources generated during the project as follows:
\begin{itemize}
    \item Project metadata description
    \item People metadata description
    \item Data Resources metadata description
\end{itemize}

This section proceeds by detailing the specific metadata categories required for cataloging and project tracking:


\subsection{Project Metadata Description}
\label{subsec:project-metadata}

This category includes metadata attributes that provide essential information about the identity, scope, and timeline of the project. The attributes are:
\begin{itemize}
    \item \texttt{Title}: encodes the name of the project.
    \item \texttt{URL}: represents the dereferenceable URL of the project.
    \item \texttt{Keywords}: includes a set of natural language keywords that capture the main theme of the project, helping to classify it.
    \item \texttt{Description}: provides a textual description of the project, summarizing its objectives and activities.
    \item \texttt{Observation}: reports some issues and characteristics of the project.
\end{itemize}
The table illustrating the project metadata can be found in Figure \ref{fig:ProjectMetadata}.

\begin{figure}[h!]
    \centering
    \includegraphics[width=\textwidth]{/home/michael/Weather-and-climate-change/Metadata/img/ProjectMetadata.png}
    \caption{Project Metadata table.}
    \label{fig:ProjectMetadata}
\end{figure}



\subsection{People Metadata Description}
\label{subsec:people-metadata}

This category includes metadata attributes related to the individuals involved in the project, which is essential for tracking and recognizing their contributions throughout the project. The table can be found in Figure \ref{fig:PeopleMetaData}.

\begin{figure}[h!]
    \centering
    \includegraphics[width=\textwidth]{/home/michael/Weather-and-climate-change/Metadata/img/PeopleMetadata.png}
    \caption{People Metadata table.}
    \label{fig:PeopleMetaData}
\end{figure}

\subsection{Data Resources Metadata Description}
\label{subsec:data-resources-metadata}

This category includes metadata attributes that describe dataset resources where we found the initial dataset from which we construct our KG. These attributes help define the dataset identity, accessibility, ownership, and technical specifications. The key attributes are:
\begin{itemize}
    \item \texttt{License}: specifies the dataset license, ensuring clarity on usage rights.
    \item \texttt{URL}: provides a dereferenceable link to access the dataset.
    \item \texttt{Publisher}, \texttt{Creator}, and \texttt{Owner}: record information about the dataset publisher, creator, and owner, respectively.
    \item \texttt{Language}: indicates the natural language(s) used in the dataset. In our project, data sources primarily use \textbf{English} and \textbf{Italian}.
    \item \texttt{Name}: provides the dataset’s name in natural language.
    \item \texttt{PublicationTimestamp} and \texttt{Version}: record the date of its publication in the catalog.
    \item \texttt{Description}: offers a textual explanation of the dataset content.
    \item \texttt{FileFormat}: encodes the dataset format, ensuring clarity on its structure and usability. In our case, the data primarily consists of \textbf{CSV} files.
\end{itemize}
The table can be found in Figure \ref{fig:DatasetMetaData}.

\begin{figure}[h!]
    \centering
    \includegraphics[width=\textwidth]{/home/michael/Weather-and-climate-change/Metadata/img/DatasetMetadata.png}
    \caption{Data Resources Metadata table.}
    \label{fig:DatasetMetaData}
\end{figure}

\noindent All files, including the metadata spreadsheets and images referenced in this section, can be found in the following \href{https://github.com/Michael-Bernasconi/Weather-and-climate-change/tree/documentation/Metadata}{GitHub directory}.