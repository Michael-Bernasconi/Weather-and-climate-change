\section{Introduction}
Reusability is one of the core principles of the Knowledge Graph (KG) development process defined by the iTelos methodology. Project documentation plays a fundamental role in enhancing the reusability of the resources handled and produced throughout the process. A clear and structured description of the resources, of the implemented processes and sub-processes, and of the evaluation activities performed at each stage enables a comprehensive understanding of the project, thereby supporting the future exploitation of its outcomes by external stakeholders.\\

\noindent This document provides a detailed report of a project developed in accordance with the iTelos methodology. The structure of the report is organized as follows:
\begin{itemize}
    \item Section 2 presents the definition of the project’s purpose and the associated information-gathering activities.
    
    \item Sections 3, 4, 5, 6, 7 describe the phases of the iTelos process and their corresponding activities, distinguishing between knowledge-layer and data-layer tasks, and include an evaluation of the produced resources with respect to their fitness for the intended purpose.
    
    \item Section 8 describes the metadata generated for all types of resources handled and produced during the execution of the iTelos process.
    
    \item Section 9 concludes the report and summarizes the identified open issues.
\end{itemize}
