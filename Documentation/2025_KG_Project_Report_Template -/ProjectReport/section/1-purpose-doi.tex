
\section{Project Design}


The project's domain of interest is defined by precise spatial and temporal boundaries, specifically selected to enable the monitoring and study of meteorological phenomena in a mountainous context. Geographically, the domain is strictly delimited to the territory of the Autonomous Province of Trento (Trentino region). This area, ranging from valley floors to alpine peaks, provides a heterogeneous environment suitable for analyzing diverse microclimatic conditions.

Temporally, the analysis encompasses a comprehensive 35-year historical window (spanning from 1990 to 2025). This duration ensures the statistical reliability needed to distinguish between transient weather variability and consolidated climate trends or anomalies over the years.

The general purpose of this project is to facilitate the systematic analysis of climate evolution, specifically focusing on the detection of long-term trends and shifting meteorological behaviors. By providing a structured Knowledge Graph (KG), the project aims to bridge the gap between raw data and semantic understanding, supporting domain experts in managing data heterogeneity. Ultimately, the KG serves as an analytical tool to gain deeper insights into weather patterns, enabling the correlation of daily observations with broader climatic variations within the Trentino region and can also help to discover anomalies in the territory.

