\section{Open Issues}
This section concludes the document by reporting final considerations on the quality of the development process and the achieved outcomes, and by describing the main issues that, due to limitations in time, resources, or data availability, remain unresolved and are therefore deferred to future project iterations.

\subsection{Adherence to Schedule and Achievement of Objectives}

The project successfully respected the initial planning, with all scheduled phases completed within the established documentation deadlines.

However, despite the significant progress achieved and the construction of a coherent and functional Knowledge Graph (KG), the final results did not fully satisfy the original objectives of the project.

In particular, the Domain of Interest (DOI) was necessarily reduced in two fundamental dimensions due to data management and scalability constraints:

\begin{itemize}
    \item \textbf{Geographical Scope:} The initial objective was to model meteorological phenomena and microclimatic variations across the entire Trentino-Alto Adige region, with a specific focus on alpine zones, valleys, and lake areas. In order to limit dataset size and ensure computational manageability, the conceptual class \texttt{Region} was redefined as \texttt{City}, restricting the analysis to ten specific locations. As a consequence, the original goal of modelling detailed microclimatic behaviors in valleys and lake districts could not be fully addressed.
    
    \item \textbf{Temporal Coverage:} For similar performance and storage-related reasons, the observation time span was reduced from the originally planned 30 years to a 15-year historical window.
\end{itemize}

\subsection{Open Issues at Project Conclusion}

The most relevant unresolved issues identified during the project, which constitute the basis for future extensions and refinements, are summarized as follows:

\begin{itemize}
    \item \textbf{Data-Driven Geographical and Temporal Constraints:}  
    A major open issue derives from the necessity to constrain both spatial and temporal dimensions of the dataset. The transition from broad geographical macro-areas to individual cities, together with the reduction of the historical time span, required a partial reformulation of the original Competency Questions (CQs). These limitations prevent a complete investigation of large-scale climate dynamics and long-term trends, thereby restricting the depth of possible inferences. Several mitigation strategies were considered, including the integration of additional local datasets and the adoption of alternative data sources with wider spatial coverage; however, implementation was not feasible within the available time frame.
    
    \item \textbf{Refinement of Complex Entity Derivation:}  
    Core Knowledge Graph entities such as \texttt{Anomaly}, \texttt{Microclimate}, \texttt{ClimateTrend}, and \texttt{Season} were generated through algorithmic derivation from raw data. Although the current implementation produced meaningful results, the underlying derivation logic requires more extensive testing, validation, and optimization. This refinement is essential to improve the accuracy, robustness, and interpretability of these entities, and to enhance the overall reliability of the KG when addressing complex analytical queries.
    
    \item \textbf{Harmonization and Consistency of Heterogeneous Data Sources:}  
    The Knowledge Graph was built by integrating data from heterogeneous sources (namely \textit{Open Data Trentino}, \textit{ilMeteo.it}, and the \textit{Historical Weather API}), each characterized by distinct schemas and formats. Although a thorough data cleaning phase was conducted, full structural harmonization and the systematic management of residual inconsistencies (arising from missing, incomplete, or outdated attributes) remain open challenges. These aspects require further validation, alignment, and standardization steps in future iterations of the model.
\end{itemize}
