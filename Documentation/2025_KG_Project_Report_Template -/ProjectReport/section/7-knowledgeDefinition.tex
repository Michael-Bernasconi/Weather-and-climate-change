\section{Knowledge Definition}
This section analyzes the Knowledge Definition phase. The definition of object and data properties was directly guided by the Purpose of the project, which required the ability to formally describe, connect, and query meteorological phenomena, climate trends, seasonal characteristics, and their relation to specific cities in the Trentino region. The Entity–Relationship (ER) model served as the conceptual backbone for this phase: each meaningful relationship identified in the ER diagram was translated into an OWL object property, preserving its semantic direction, cardinality, and intended meaning within the domain.

Attributes associated with entities in the ER schema were transformed into data properties, for which appropriate XSD datatypes were selected to ensure semantic correctness, validation, and interoperability. The Data Layer played a crucial role in consolidating and validating these modeling decisions. Whenever discrepancies or inconsistencies emerged between the ER structure and the real datasets, the ontology was shaped to remain both faithful to the conceptual model and compatible with the actual data. Through this process, the resulting ontology maintains strong coherence across Purpose, ER, and Data Layer—allowing semantically meaningful integration, querying, and reasoning over the heterogeneous climate and weather data sources involved in the project. 

\subsection{Object properties}
\begin{figure}[h]
    \centering
    \includegraphics[width=0.5\textwidth]{../../../../Phase 3 - Knowledge Definition/img/object_properties.png}
    \caption{Object properties defined for the ontology.}
    \label{fig:object_properties}
\end{figure}

The result of the object properties definition is shown in Figure \ref{fig:object_properties}. To represent information such as temperature, precipitation, and humidity—each requiring a set of values (e.g., minimum, mean, and maximum)—a hierarchy was defined in which each observed aspect is modeled as a sub-property of \texttt{has\_phenomenon\_information}. This representation relies on subclasses of \textit{WeatherPhenomenon}, which are already defined in the Home Weather Ontology. This approach allows key characteristics of each phenomenon (e.g., precipitation measured in millimeters) to be consistently represented.

To define the location of each \textit{WeatherStation}, an object property already defined in the Home Weather Location ontology was reused to relate it to the concept of \textit{SpatialThing}, which includes predefined attributes such as altitude, longitude, and latitude.

Another aspect modeled as an object property is the temperature range associated with a microclimate. For this purpose, the concept of \textit{Range}, including upper and lower bound attributes, was introduced. This concept was modeled as an eType to ensure potential reusability in other contexts. The object property \texttt{has\_temperature\_range} was then defined to establish the relationship between the two eTypes.

Each object property was defined by specifying its domain, range, and semantic direction, ensuring alignment with the ER model and consistency with OWL best practices. Functional characteristics were assigned when appropriate (e.g., each \textit{YearSeason} is observed in exactly one \textit{City}).

\subsection{Data properties}
\begin{figure}[h]
    \centering
    \includegraphics[width=0.5\textwidth]{../../../../Phase 3 - Knowledge Definition/img/data_properties.png}
    \caption{Data properties defined for the ontology.}
    \label{fig:data_properties}
\end{figure}

The result of the data properties definition phase is shown in Figure \ref{fig:data_properties}. Data properties related to spatial position, such as latitude, longitude, and altitude, were already defined in the Home Weather Ontology. The remaining data properties were defined based on the ER model of the domain and the object properties previously introduced.

One modification with respect to the ER model concerns temporal attributes: date and time, originally defined as separate attributes, were merged into a single attribute of type \texttt{datetime}.

Another relevant aspect modeled as a data property is the \textit{type} attribute for anomalies and microclimates. This attribute was modeled as a data property of type \texttt{string}, rather than as an object property linked to a dedicated type class, in order to allow greater flexibility in defining and extending possible values.

All data properties were assigned appropriate XSD datatypes to guarantee precise validation and interoperability. Mandatory attributes (e.g., \textit{detectionDate} for \textit{Anomaly}) were distinguished from optional ones, reflecting both the ER model and the structure of the available datasets.

To clarify the overall Knowledge Definition process, three representative examples are presented—one for each modeling activity: entity definition, object property definition, and data property definition.

\subsection{Entity Definition Example: Weather Station}
During the entity definition phase, the core concepts relevant to the project Purpose and the ER model were identified. As part of this process, existing concepts in the Home Weather Ontology (HWO) were analyzed. Since HWO includes the concept \textit{WeatherReportSource}, which semantically aligns with the notion of a physical station producing weather observations, the ontology was extended by defining \textit{WeatherStation} as a subclass of \textit{WeatherReportSource}.

This choice enables the reuse of established semantic structures while tailoring the concept to the specific domain requirements. A descriptive comment was added to clarify the meaning and expected role of \textit{WeatherStation} within the ontology. The resulting definition is shown in Figure \ref{fig:entity_example}.

\begin{figure}[h]
    \centering
    \includegraphics[width=0.5\textwidth]{../../../../Phase 3 - Knowledge Definition/img/entity_example.png}
    \caption{Example of entity: the \textit{WeatherStation} entity.}
    \label{fig:entity_example}
\end{figure}

\subsection{Object Property Definition Example: has\_wind\_information}
Once the entities were defined, the creation of object properties was carried out to model relationships between eTypes. As an illustrative example, the relationship between \textit{WeatherReport} and \textit{Wind} is considered.

Since each weather report includes a complete set of wind-related measurements, the object property \texttt{has\_wind\_information} was introduced. The domain of this property is \textit{WeatherReport}, while its range is \textit{Wind}. This choice preserves the semantic direction (a report includes wind information) and ensures compatibility with the structure of the Data Layer, where all wind attributes originate from the same weather report. Figure \ref{fig:object_property_example} shows the final definition.

\begin{figure}[h]
    \centering
    \includegraphics[width=0.5\textwidth]{../../../../Phase 3 - Knowledge Definition/img/object_property_example.png}
    \caption{Example of object property linking \textit{WeatherReport} to \textit{Wind}.}
    \label{fig:object_property_example}
\end{figure}

\subsection{Data Property Definition Example: WeatherStation code}
Finally, data properties were defined. When possible, datatype properties already defined in the W3.org ontologies adopted by HWO—such as those related to geographic coordinates (\textit{Point}) or wind characteristics (\textit{Wind})—were reused. For concepts introduced in the ontology, new data properties were defined accordingly.

For example, each \textit{WeatherStation} requires a unique identifier, represented in the datasets as an alphanumeric code. This identifier was modeled as a datatype property with domain \textit{WeatherStation} and range \texttt{xsd:string}. Figure \ref{fig:data_property_example} shows the final structure of this property.

\begin{figure}[h]
    \centering
    \includegraphics[width=0.5\textwidth]{../../../../Phase 3 - Knowledge Definition/img/data_property_example.png}
    \caption{Example of data property modeling the unique code of a \textit{WeatherStation}.}
    \label{fig:data_property_example}
\end{figure}

These three examples illustrate the methodology applied throughout the entire Knowledge Definition phase: starting from the ER model and guided by the project Purpose and the available datasets, entities, object properties, and data properties were defined to form a coherent and semantically rich ontology.

In conclusion, the final result of this phase is the \href{https://github.com/Michael-Bernasconi/Weather-and-climate-change/blob/50495365ff2c8e0d8fb5170f74a65995a37dc640/Phase%203%20-%20Knowledge%20Definition/knowledge_definition.rdf}{ontology file}. No modifications were required to the previously defined competency questions or ER model, as all aspects could be fully modeled within the Knowledge Definition phase.
