\section{Purpose Definition}
The iTelos methodology proposes a systematic approach designed to simplify and reduce the effort required to build Knowledge Graphs (KGs), focusing on the specific purpose indicated by the end user. This section provides a detailed overview of the first phase of the methodology.

\subsection{Purpose Formalization}
In this phase, the informal purpose is structured and formalized to guide the development of the Knowledge Graph. Purpose formalization includes the specification of the Domain of Interest, the identification of the main concepts (Concept Identification), the definition of usage scenarios and personas, the formulation of the Competency Questions (CQs) that the Knowledge Graph must be able to answer, and the definition of the conceptual model (ER Model Definition). This step ensures that the design of the KG is aligned with user requirements and provides a clear and consistent framework for the subsequent modeling and implementation phases.
\subsubsection{Informal Purpose}
The purpose of this project is to build a Knowledge Graph (KG) that models the meteorological facilities in the territory and, consequently, the climate and potential climate change in Trentino. The KG will organize information in a structured and accessible way, allowing it to answer precise user queries, such as identifying the locations of weather stations, analyzing temperature and rainfall over recent years, pinpointing areas with the highest temperature increases, and detecting signs of climate change based on historical data.

\subsubsection{Domain of Interest}
The Domain of Interest (DoI) for this project is the Trentino region in 2025, with a particular focus on meteorological phenomena and climate change. The Trentino region exhibits a wide variety of microclimates and weather conditions, ranging from high alpine areas to valleys and lakes, making it an ideal natural laboratory for climate study. The project’s geographic scope covers the entire region, including weather stations, climate sensors, and historical data, enabling a comprehensive analysis of climate patterns.\\
Key features of the domain include:
\begin{itemize}
    \item \textit{Meteorological Monitoring}: Data on temperature, precipitation, humidity, wind, and other variables measured by weather stations distributed throughout the territory.
    \item \textit{Climate Change Monitoring}: Analysis of historical trends and detection of climate variation signals, such as increases in average temperature, changes in precipitation patterns, and local climatic anomalies.
\end{itemize}


\subsubsection{Scenario}
This section presents several usage scenarios, describing the different aspects considered by the project’s purpose.
\begin{enumerate}
    \item \textbf{Maria} and her boyfriend, two local researchers, are analyzing precipitation trends in Trentino over the past ten years. They want to identify areas where rainfall has significantly increased or decreased to understand potential impacts on agricultural and forested areas. They use the Knowledge Graph to retrieve historical data from multiple weather stations and compare time series.
    \item \textbf{Giulia}, a climatology student, wants to study the microclimates present in the different valleys of Trentino. She needs access to temperature, humidity, and wind data collected by sensors across the territory to analyze climate variations between alpine and lake areas.
    \item \textbf{Alessandro}, responsible for civil protection, is monitoring climatic anomalies in real time. He aims to quickly identify areas with unusual temperatures or precipitation to plan preventive interventions against landslides or hydrogeological risks.
    \item \textbf{Francesca} and her group of students want to study the impact of climate change on the seasons in Trentino. They need to compare historical data with current measurements to understand the evolution of climatic phenomena, such as delayed snowfall or increased heatwaves.
    \item \textbf{Marco}, a weather enthusiast, uses the Knowledge Graph to explore long-term trends in temperature and precipitation, identify areas with the highest temperature increases, and better understand the signals of climate change in the Trentino region.
\end{enumerate}

\subsubsection{Personas}
This section defines a set of real users acting within the previously described scenarios.
\begin{enumerate}
    \item \textbf{Maria Bianchi}, 35 years old, a local researcher passionate about meteorology. She is interested in analyzing precipitation trends and the impact of climate change on the territory.
    \item \textbf{Giulia Ferrari}, 23 years old, a climatology student. She enjoys studying microclimates and climate variations between alpine and lake areas.
    \item \textbf{Alessandro Rossi}, 40 years old, responsible for civil protection. He monitors climatic anomalies in real time to prevent landslides and hydrogeological risks.
    \item \textbf{Francesca Romano}, 22 years old, an Erasmus student. She is interested in comparing historical and current data to study the evolution of seasonal climate phenomena.
    \item \textbf{Marco Ricci}, 28 years old, a weather enthusiast. He likes exploring long-term temperature and precipitation trends to better understand the signals of climate change in the Trentino region
\end{enumerate}



\subsubsection{Competency Questions (CQs)}
\begin{tabular}{|p{5cm}|p{1cm}|p{10cm}|}
\hline
\textbf{Person} & \textbf{N.o.} & \textbf{Competency Question (CQ)} \\
\hline
Maria Bianchi & 1.1 & Which areas of Trentino have experienced significant increase in annual rainfall over the past decade? \\ Maria Bianchi & 1.2 & Which areas have shown a consistent decrease in precipitation over the past decade? \\ Maria Bianchi & 1.3 & How has the average monthly rainfall evolved in each valley or municipality of Trentino over the last ten years? \\ Maria Bianchi & 1.4 & Are there correlations between changes in rainfall and altitude or proximity to forests and agricultural land?\\ Maria Bianchi & 1.5 & Which weather stations have the most complete historical data on precipitation in Trentino? \\
\hline
Giulia Ferrari & 2.1 & What are the typical temperature ranges and humidity levels in alphine valleys compared to lake areas? \\ Giulia Ferrari & 2.2 & Which areas show microclimatic differences despite geographical proximity? \\ Giulia Ferrari & 2.3 & How does wind speed and direction vary between the Adige Valley and surrounding mountains areas? \\ Giulia Ferrari & 2.4 & Are there correlations between altitude and average annual temperature or humidity? \\ Giulia Ferrari & 2.5 & How stable are microclimates across seasons (winter vs. summer) in different valleys? \\ Giulia Ferrari & 2.6 & Which valleys show the most distinctive microclimatic patterns according to sensor data? \\
\hline
Alessandro Rossi & 3.1 & Which areas of Trentino are currently showing unusually high or low temperatures compared to historical averages? \\ Alessandro Rossi & 3.2 & Are there real-time alerts for abnormal precipitation or snowmelt that could indicate flood risks? \\ Alessandro Rossi & 3.3 & Which zones are currently under potential hydrogeological risk due to recent heavy rainfall? \\ Alessandro Rossi & 3.4 & How do current weather anomalies compare to past extreme events recorded in the KG? \\ Alessandro Rossi & 3.5 & Can the KG highlight areas with recurring climatic anomalies over multiple years? \\ Alessandro Rossi & 3.6 & Which meteorological stations are currently reporting anomalies in temperature or precipitation beyond expected thresholds?\\
\hline
Francesca Romano & 4.1 & How have the average start and end dates of each season changed in Trentino over the past 30 years?\\ Francesca Romano & 4.2 & Has the timing or duration of snowfall periods shifted over time? \\ Francesca Romano & 4.3 & Are heat waves occurring more frequently or lasting longer than in the past? \\ 
\hline
\end{tabular}
\begin{tabular}{|p{5cm}|p{1cm}|p{10cm}|}
\hline
Francesca Romano & 4.4 & How does current spring temperature compare to historical averages from the 1980s and 1990s? \\ Francesca Romano & 4.5 & Which areas show the most significant seasonal shifts (e.g., warmer winters, delayed autumn)? \\ Francesca Romano & 4.6 & How has average precipitation in summer and winter evolved over time?\\
\hline
Marco Ricci & 5.1 & Which areas of Trentino gave recorder the highest temperature increases over the past 50 years? \\ Marco Ricci & 5.2 & How have long-term temperature and precipitation trends evolved across different valleys? \\ Marco Ricci & 5.3 & What are the clearest signals of climate change (e.g., rising temperatures, changing rainfall patterns) in the KG data? \\ Marco Ricci & 5.4 & Are there locations showing evidence of both increased temperature and decreased precipitation? \\ Marco Ricci & 5.5 & Can the KG visualize how average annual temperatures have evolved decade by decade? \\ Marco Ricci & 5.6 & Which weather stations show the most evident long-term warming trends? \\
\hline
\end{tabular}
\subsubsection{Concept Identification}
Concept identification aims to identify which are the main entities and components relevant to the defined purpose. In Figure \ref{fig:purpose_formalization_sheet} is reported the Purpose formalization sheet used to collect the main entities in the project.
\begin{figure}[h]
    \centering
    \includegraphics[width=1\textwidth]{../knowdive-files/purpose_formalization_sheet.png}
    \caption{Purpose Formalization Sheet - Concept Identification}
    \label{fig:purpose_formalization_sheet}
\end{figure}
\subsubsection{ER model definition}
The purpose of the Knowledge Graph is to provide users with comprehensive access to meteorological information over time.
To achieve this, an Entity–Relationship (ER) model was designed according to the Competency Questions (CQs) defined in the previous section, ensuring that all user queries can be expressed and satisfied.
The central entity of the model is \textit{WeatherReport}, which represents the set of meteorological observations collected over time.
A specific focus is given to \textit{PrecipitationReport}, a component of the \textit{WeatherReport} entity that captures precipitation-related data relevant to several queries.
Spatial aspects are represented through the \textit{WeatherStation} and \textit{Region} entities, which enable location-based analyses and support the spatial dimension of the CQs.
Additional entities, such as \textit{ClimateTrend}, \textit{Season}, \textit{Anomaly}, and \textit{Microclimate}, enrich the model by representing temporal patterns, climatic variations, and exceptional phenomena, as required by the different user scenarios.
Figure \ref{fig:ER_model} illustrates the overall structure of the ER model, which integrates temporal, spatial, and climatic perspectives.
\begin{figure}[h]
    \centering
    \includegraphics[width=1\textwidth]{../../../../Phase 1 - Purpose Definition/er_model_modified.png}
    \caption{ER Model for Meteorological KG}
    \label{fig:ER_model}
\end{figure}
\subsubsection{Report}
In this section, the purpose of the Knowledge Graph project has been formally defined.
Personas and scenarios were developed to capture different user perspectives and information needs, with particular attention to the historical analysis and study of past climatic phenomena.
A set of Competency Questions (CQs) was then formulated to specify the queries that the Knowledge Graph should be able to answer in order to address these needs.
Finally, an Entity–Relationship (ER) model was designed to identify the main entities and relationships required to represent the domain of interest and to support the answering of the CQs.

The ER has to be evaluated in the next section, in order to verify its completeness and correctness with respect to the data sources that will be used.
\\ 





\subsection{Information Gathering}
In this section, it's presented an overview about available information for this project. The information are used also to define knowledge concepts and to retrive data for populating the Knowledge Graph.
\subsubsection{Knowledge Sources}
The main knowledge sources identified for this project are:
\begin{itemize}
    \item \textbf{\href{https://lov.linkeddata.es/dataset/lov/vocabs/hw}{Home Weather Ontology}} is one of the main reference resources used in this project. 
    It provides an ontology for describing weather phenomena and external environmental conditions. 
    The ontology consists of 106 classes and 33 properties covering various aspects of meteorological observations, such as temperature, humidity, atmospheric pressure, wind, and precipitation. 
    In the context of this project, the ontology has been used to retrieve the terminology required for data integration and to define the core entities, their semantic scope, and usage within the Knowledge Graph. 
    Thanks to its well-structured design, the Home Weather Ontology serves as a solid foundation for modeling meteorological information collected from weather stations across the Trentino region.
    \item \textbf{\href{https://schema.org/docs/full.htm}{Schema.org}} is a collaborative and community-driven initiative with the mission to create, maintain, and promote schemas for structured data on the Web. The schemas consist of a set of types, each associated with a set of properties, organized within a semantic hierarchy.
    In the context of this project, Schema.org is used to complete the list of types relevant to the Domain of Interest (DoI), particularly for the description of territorial and organizational entities such as WeatherStation and Region, ensuring interoperability and alignment with Semantic Web standards.
    \item \textbf{DCAT}: is a W3C-recommended ontology designed to facilitate the publication, discovery, and interoperability of datasets on the Web. It provides a standardized structure for describing data catalogs, datasets, and their distributions, including essential metadata such as creators, licenses, temporal and spatial coverage, and formats. In the context of this project, DCAT is employed to formally describe the meteorological datasets collected from different sources, such as Open Data Trentino and external climate repositories, ensuring that each dataset is traceable, well-documented, and aligned with FAIR principles (Findable, Accessible, Interoperable, Reusable). Through the integration of DCAT classes such as dcat:Dataset and dcat:Distribution, the Knowledge Graph supports consistent documentation of data provenance, licensing, and accessibility, thus enhancing the transparency and reusability of the meteorological and climatic information modeled in the Trentino domain.
\end{itemize}
\subsubsection{Data Sources}
The main data sources identified for this project are:
\begin{itemize}
\item \textbf{\href{https://dati.trentino.it/dataset/observations-site-list}{Open Data Trentino}}: provides a comprehensive list of meteorological stations in the Trentino region. The dataset includes information such as station location, altitude, and available measurements, which can be used to analyze local weather conditions over time. The data is freely accessible and can be integrated into applications for environmental monitoring and weather analysis.
\item \textbf{\href{https://www.ilmeteo.it/portale/archivio-meteo/Trento}{ilMeteo.it}}:  gives access to historical weather data for Trento from 1973 to 2025. For each year, there are historical data for each day of months. The main attributes for the weather reports are:
    \begin{itemize}
        \item Average temperature (float)
        \item Minimum temperature (float)
        \item Maximum temperature (float)
        \item Precipitation (float)
        \item Umidity rate (float)
        \item Wind speed max (float)
        \item Wind speed average (float)
        \item Metereological phenomena (string)
    \end{itemize}
    All the attributes indicated above are usefull to answer the Competency Questions (CQs) defined in the previous section.
    \item \textbf{\href{https://open-meteo.com/en/docs/historical-weather-api}{Historical Weather API}}: it is based on reanalysis dataset and use of weather station, aircraft, buoy, radar and satellite observations to create a record of past weather conditions. The data is available for any location worldwide. The datasets used are: ECMWF IFS, ERA5, ERA5-Land, ERA5-Ensemble, CERRA, ECMWF IFS Assimilation Long Window. There is a set of meteorological variables available, both in hourly and daily granularity. The main attributes usefull for our project are: 
    \begin{itemize}
        \item \textbf{Weather Code}: is a numerical  identifier used to identify weather phenomena. In our case, it's useful for queries about analysis of weather phenomena in time and also to rilevate anomalies.
        \item \textbf{Mean, Maximum and Minimum temperature}: these data, like for ilMeteo.it source, are relevant for the majority of scenarios: in particular, for Scenario 2 as it helps to analyse microclimates, for Scenario 3 to track anomalies related to temperatures, Scenario 4 to track temperature changes and Scenario 5 for temperature long term trends. 
        \item \textbf{Precipitation, Rain and Snowfall sum and Precipitation Hours}
    \end{itemize}
\end{itemize}